% ==============================================================================
% Modelo para Monografia de Projeto de Graduação - Ciência da Computação (em Português)
% Prof. Vítor E. Silva Souza - NEMO/UFES :: DI/UFES :: PPGI/UFES
% Com adaptações feitas pelo Colegiado de Engenharia de Computação / CT / UFES e pela profª. Monalessa Perini Barcellos (NEMO/UFES).
%
%
% Baseado em abtex2-modelo-trabalho-academico.tex, v-1.9.2 laurocesar
% Copyright 2012-2014 by abnTeX2 group at http://abntex2.googlecode.com/ 
%
% This work may be distributed and/or modified under the conditions of the LaTeX 
% Project Public License, either version 1.3 of this license or (at your option) 
% any later version. The latest version of this license is in
% http://www.latex-project.org/lppl.txt.
%
% IMPORTANTE:
% Instruções encontram-se espalhadas pelo documento. Para facilitar sua leitura,
% tais instruções são precedidas por (*) -- utilize a função localizar do seu
% editor para passar por todas elas.
% ==============================================================================

% Usa o estilo abntex2, configurando detalhes de formatação e hifenização.
\documentclass[
	12pt,				% Tamanho da fonte.
	openright,			% Capítulos começam em página ímpar (insere página vazia caso preciso).
	oneside,			% Para impressão em verso e anverso. Oposto a oneside.
	a4paper,			% Tamanho do papel.
	english,			% Idioma adicional para hifenização.
	french,				% Idioma adicional para hifenização.
	spanish,			% Idioma adicional para hifenização.
	brazil				% O último idioma é o principal do documento.
	]{abntex2}



%%% Importação de pacotes. %%%

% Conserta o erro "No room for a new \count". 
% O comando \reserveinserts deve ser comentado ou não, dependendo da versão do LaTeX.
\usepackage{etex}
%\reserveinserts{28}

% Usa a fonte Latin Modern.
\usepackage{lmodern}

% Seleção de códigos de fonte.
\usepackage[T1]{fontenc}

% Codificação do documento em Unicode.
\usepackage[utf8]{inputenc}

% Usado pela ficha catalográfica.
\usepackage{lastpage}

% Indenta o primeiro parágrafo de cada seção.
\usepackage{indentfirst}

% Controle das cores.
\usepackage[usenames,dvipsnames]{xcolor}

% Inclusão de gráficos.
\usepackage{graphicx}

% Tabularx package: melhor controle de leiaute de tabelas.
\usepackage{tabularx}

% Inclusão de páginas em PDF diretamente no documento (para uso nos apêndices).
\usepackage{pdfpages}

% Para melhorias de justificação.
\usepackage{microtype}

% Citações padrão ABNT.
\usepackage[brazilian,hyperpageref]{backref}
\usepackage[alf]{abntex2cite}	
\renewcommand{\backrefpagesname}{Citado na(s) página(s):~}		% Usado sem a opção hyperpageref de backref.
\renewcommand{\backref}{}										% Texto padrão antes do número das páginas.
\renewcommand*{\backrefalt}[4]{									% Define os textos da citação.
	\ifcase #1
		Nenhuma citação no texto.
	\or
		Citado na página #2.
	\else
		Citado #1 vezes nas páginas #2.
	\fi}

% \rm is deprecated and should not be used in a LaTeX2e document
% http://tex.stackexchange.com/questions/151897/always-textrm-never-rm-a-counterexample
\renewcommand{\rm}{\textrm}

% Inclusão de símbolos não padrão.
\usepackage{amssymb}
\usepackage{eurosym}

% Para utilizar \eqref para referenciar equações.
\usepackage{amsmath}

% Permite mostrar figuras muito largas em modo paisagem com \begin{sidewaysfigure} ao invés de \begin{figure}.
\usepackage{rotating}

% Permite customizar listas enumeradas/com marcadores.
\usepackage{enumitem}

% Permite inserir hiperlinks com \url{}.
\usepackage{bigfoot}
\usepackage{hyperref}

% Permite usar o comando \hl{} para evidenciar texto com fundo amarelo. Útil para chamar atenção a itens a fazer.
\usepackage{soulutf8}

% Colorinlistoftodos package: to insert colored comments so authors can collaborate on the content.
\usepackage[colorinlistoftodos, textwidth=20mm, textsize=footnotesize]{todonotes}
\newcommand{\aluno}[1]{\todo[author=\textbf{Aluno},color=green!30,caption={},inline]{#1}}
\newcommand{\professor}[1]{\todo[author=\textbf{Professor},color=red!30,caption={},inline]{#1}}

% Permite inserir espaço em branco condicional (incluído no texto final só se necessário) em macros.
\usepackage{xspace}

% Permite incluir listagens de código com o comando \lstinputlisting{}.
\usepackage{listings}
\usepackage{caption}
\DeclareCaptionFont{white}{\color{white}}
\DeclareCaptionFormat{listing}{\colorbox{gray}{\parbox{\textwidth}{#1#2#3}}}
\captionsetup[lstlisting]{format=listing,labelfont=white,textfont=white}
\renewcommand{\lstlistingname}{Listagem}
\definecolor{mygray}{rgb}{0.5,0.5,0.5}
\lstset{
	basicstyle=\scriptsize,
	breaklines=true,
	numbers=left,
	numbersep=5pt,
	numberstyle=\tiny\color{mygray}, 
	rulecolor=\color{black},
	showstringspaces=false,
	tabsize=2,
    inputencoding=utf8,
    extendedchars=true,
    literate=%
    {é}{{\'{e}}}1
    {è}{{\`{e}}}1
    {ê}{{\^{e}}}1
    {ë}{{\¨{e}}}1
    {É}{{\'{E}}}1
    {Ê}{{\^{E}}}1
    {û}{{\^{u}}}1
    {ù}{{\`{u}}}1
    {â}{{\^{a}}}1
    {à}{{\`{a}}}1
    {á}{{\'{a}}}1
    {ã}{{\~{a}}}1
    {Á}{{\'{A}}}1
    {Â}{{\^{A}}}1
    {Ã}{{\~{A}}}1
    {ç}{{\c{c}}}1
    {Ç}{{\c{C}}}1
    {õ}{{\~{o}}}1
    {ó}{{\'{o}}}1
    {ô}{{\^{o}}}1
    {Õ}{{\~{O}}}1
    {Ó}{{\'{O}}}1
    {Ô}{{\^{O}}}1
    {î}{{\^{i}}}1
    {Î}{{\^{I}}}1
    {í}{{\'{i}}}1
    {Í}{{\~{Í}}}1
}



%%% Definição de variáveis. %%%

% (*) Substituir os textos abaixo com as informações apropriadas.
\titulo{Título do Trabalho}
\autor{Nome do Aluno}
\local{Vitória, ES}
\data{\the\year}
\orientador{Nome do Orientador}
\coorientador{Nome do Co-orientador}
\instituicao{
  Universidade Federal do Espírito Santo -- UFES
  \par
  Centro Tecnológico
  \par
  Colegiado do Curso de Ciência da Computação}
\tipotrabalho{Monografia (PG)}

% Preâmbulo (tipo do trabalho, objetivo, nome da instituição, área de concentração, etc.).
\preambulo{Monografia apresentada ao Curso de Ciência da Computação do Centro Tecnológico da Universidade Federal do Espírito Santo, como requisito parcial para obtenção do Grau de Bacharel em Ciência da Computação.}

% Macros específicas do trabalho.
% (*) Inclua aqui termos que são utilizados muitas vezes e que demandam formatação especial.
% Os exemplos abaixo incluem i* (substituindo o asterisco por uma estrela) e Java com TM em superscript.
% Use sempre \xspace para que o LaTeX inclua espaço em branco após a macro somente quando necessário.
\newcommand{\istar}{\textit{i}$^\star$\xspace}
\newcommand{\java}{Java\texttrademark\xspace}
\newcommand{\latex}{\LaTeX\xspace}




%%% Configurações finais de aparência. %%%

% Altera o aspecto da cor azul.
\definecolor{blue}{RGB}{41,5,195}

% Informações do PDF.
\makeatletter
\hypersetup{
	pdftitle={\@title}, 
	pdfauthor={\@author},
	pdfsubject={\imprimirpreambulo},
	pdfcreator={LaTeX with abnTeX2},
	pdfkeywords={abnt}{latex}{abntex}{abntex2}{trabalho acadêmico}, 
	colorlinks=true,				% Colore os links (ao invés de usar caixas).
	linkcolor=blue,					% Cor dos links.
	citecolor=blue,					% Cor dos links na bibliografia.
	filecolor=magenta,				% Cor dos links de arquivo.
	urlcolor=blue,					% Cor das URLs.
	bookmarksdepth=4
}
\makeatother

% Espaçamentos entre linhas e parágrafos.
\setlength{\parindent}{1.3cm}
\setlength{\parskip}{0.2cm}



%%% Páginas iniciais do documento: capa, folha de rosto, ficha, resumo, tabelas, etc. %%%

% Compila o índice.
\makeindex

% Inicia o documento.
\begin{document}

% Brasão da instituição.
\begin{figure}
	\centering
	\includegraphics[width=.20\textwidth]{figuras/brasao.jpg} 
	\label{fig-brasao}
\end{figure}

\begin{center}
	\textbf{\textsf{UNIVERSIDADE FEDERAL DO ESPÍRITO SANTO}}
	
	\textbf{\textsf{CENTRO TECNOLÓGICO}}
	
	\textbf{\textsf{COLEGIADO DO CURSO DE CIÊNCIA DA COMPUTAÇÃO}}
	
	\large{\textbf{\textsf{  }}}
	
	\large{\textbf{\textsf{  }}}
\end{center}

% Retira espaço extra obsoleto entre as frases.
\frenchspacing

% Capa do trabalho.
\imprimircapa

% Folha de rosto (o * indica que haverá a ficha bibliográfica).
\imprimirfolhaderosto*


% Ficha catalográfica.
% (*) Escolher entre as versões de ficha catalográfica abaixo (comente aquela que não quiser usar).

% Versão 1: caso a biblioteca da sua universidade lhe forneça um PDF (adequar o nome do arquivo).
% \begin{fichacatalografica}
%     \includepdf{include-fichacatalografica.pdf}
% \end{fichacatalografica}

% Versão 2: caso você tenha que inserir sua própria ficha catalográfica.
% (*) Neste caso, preencher palavras-chave e adicione co-orientador (se houver).
\begin{fichacatalografica}
	\vspace*{\fill}
	\hrule
	\begin{center}
	\begin{minipage}[c]{12.5cm}
	
	\imprimirautor
	
	\hspace{0.5cm} \imprimirtitulo  / \imprimirautor. --
	\imprimirlocal, \imprimirdata-
	
	\hspace{0.5cm} \pageref{LastPage} p. : il. (algumas color.) ; 30 cm.\\
	
	\hspace{0.5cm} \imprimirorientadorRotulo~\imprimirorientador\\
	
	\hspace{0.5cm}
	\parbox[t]{\textwidth}{\imprimirtipotrabalho~--~\imprimirinstituicao,
	\imprimirdata.}\\
	
	\hspace{0.5cm}
		1. Palavra-chave1.
		2. Palavra-chave2.
		I. Souza, Vítor Estêvão Silva.
		II. Universidade Federal do Espírito Santo.
		IV. \imprimirtitulo \\ 			
	
	\hspace{8.75cm} CDU 02:141:005.7\\
	
	\end{minipage}
	\end{center}
	\hrule
\end{fichacatalografica}


% Folha de aprovação.
% (*) Escolher entre as versões de ficha catalográfica abaixo (comente aquela que não quiser usar).

% Versão 1: cópia digitalizada da folha de aprovação assinada pela banca.
% \includepdf{include-folhadeaprovacao.pdf}

% Versão 2: folha de aprovação em branco.
% (*) Ajustar a data e os nomes dos participantes da banca.
\begin{folhadeaprovacao}
  \begin{center}
    {\ABNTEXchapterfont\large\imprimirautor}
    \vspace*{\fill}\vspace*{\fill}
    \begin{center}
      \ABNTEXchapterfont\bfseries\Large\imprimirtitulo
    \end{center}
    \vspace*{\fill}
    \hspace{.45\textwidth}
    \begin{minipage}{.5\textwidth}
        \imprimirpreambulo
    \end{minipage}%
    \vspace*{\fill}
   \end{center}
   Trabalho aprovado. \imprimirlocal, (dia) de (mês) de (ano):
   \assinatura{\textbf{\imprimirorientador} \\ Orientador} 
   \assinatura{\textbf{Nome do Membro da Banca} \\ Nome da Instituição}
   \assinatura{\textbf{Nome do Membro da Banca} \\ Nome da Instituição}
   %\assinatura{\textbf{Nome do Membro da Banca} \\ Nome da Instituição}
   %\assinatura{\textbf{Nome do Membro da Banca} \\ Nome da Instituição}
   \begin{center}
    \vspace*{0.5cm}
    {\large\imprimirlocal}
    \par
    {\large\imprimirdata}
    \vspace*{1cm}
  \end{center}  
\end{folhadeaprovacao}


% Dedicatória.
% (*) Escrever dedicatória ou remover/comentar seção.
\begin{dedicatoria}
   \vspace*{\fill}
   \centering
   \noindent
   \textit{Lorem ipsum dolor sit amet, consectetur adipiscing elit. Duis malesuada laoreet leo at interdum. Nullam neque eros, dignissim sed ipsum sed, sagittis laoreet nisi.} \vspace*{\fill}
\end{dedicatoria}


% Agradecimentos.
% (*) Escrever agradecimentos ou remover/comentar seção.
\begin{agradecimentos}
Lorem ipsum dolor sit amet, consectetur adipiscing elit. Duis malesuada laoreet leo at interdum. Nullam neque eros, dignissim sed ipsum sed, sagittis laoreet nisi. Duis a pulvinar nisl. Aenean varius nisl eu magna facilisis porttitor. Cum sociis natoque penatibus et magnis dis parturient montes, nascetur ridiculus mus. Ut mattis tortor nisi, facilisis molestie arcu hendrerit sed. Donec placerat velit at odio dignissim luctus. Suspendisse potenti. Integer tristique mattis arcu, ut venenatis nulla tempor non. Donec at tincidunt nulla. Cras ac dignissim neque. Morbi in odio nulla. Donec posuere sem finibus, auctor nisl eu, posuere nisl. Duis sit amet neque id massa vehicula commodo dapibus eu elit. Sed nec leo eu sem viverra aliquet. Nam at nunc nec massa rutrum aliquam sed ac ante.

Vivamus nec quam iaculis, tempus ipsum eu, cursus ante. Phasellus cursus euismod auctor. Fusce luctus mauris id tortor cursus, volutpat cursus lacus ornare. Proin tristique metus sed est semper, id finibus neque efficitur. Cras venenatis augue ac venenatis mollis. Maecenas nec tellus quis libero consequat suscipit. Aliquam enim leo, pretium non elementum sit amet, vestibulum ut diam. Maecenas vitae diam ligula.

Fusce ac pretium leo, in convallis augue. Mauris pulvinar elit rhoncus velit auctor finibus. Praesent et commodo est, eu luctus arcu. Vivamus ut porta tortor, eget facilisis ex. Nunc aliquet tristique mauris id sollicitudin. Donec quis commodo metus, sit amet accumsan nibh. Cum sociis natoque penatibus et magnis dis parturient montes, nascetur ridiculus mus.
\end{agradecimentos}


% Epígrafe.
% (*) Escrever epígrafe ou remover/comentar seção.
\begin{epigrafe}
    \vspace*{\fill}
	\begin{flushright}
		\textit{``Lorem ipsum dolor sit amet, consectetur adipiscing elit. \\
		Duis malesuada laoreet leo at interdum. Nullam neque eros, dignissim \\
		sed ipsum sed, sagittis laoreet nisi.\\
		(Lipsum generator)}
	\end{flushright}
\end{epigrafe}


% Resumo em português.
% (*) Escrever resumo e palavras-chave.
\setlength{\absparsep}{18pt}
\begin{resumo}
Lorem ipsum dolor sit amet, consectetur adipiscing elit. Duis malesuada laoreet leo at interdum. Nullam neque eros, dignissim sed ipsum sed, sagittis laoreet nisi. Duis a pulvinar nisl. Aenean varius nisl eu magna facilisis porttitor. Cum sociis natoque penatibus et magnis dis parturient montes, nascetur ridiculus mus. Ut mattis tortor nisi, facilisis molestie arcu hendrerit sed. Donec placerat velit at odio dignissim luctus. Suspendisse potenti. Integer tristique mattis arcu, ut venenatis nulla tempor non. Donec at tincidunt nulla. Cras ac dignissim neque. Morbi in odio nulla. Donec posuere sem finibus, auctor nisl eu, posuere nisl. Duis sit amet neque id massa vehicula commodo dapibus eu elit. Sed nec leo eu sem viverra aliquet. Nam at nunc nec massa rutrum aliquam sed ac ante.

Vivamus nec quam iaculis, tempus ipsum eu, cursus ante. Phasellus cursus euismod auctor. Fusce luctus mauris id tortor cursus, volutpat cursus lacus ornare. Proin tristique metus sed est semper, id finibus neque efficitur. Cras venenatis augue ac venenatis mollis. Maecenas nec tellus quis libero consequat suscipit. Aliquam enim leo, pretium non elementum sit amet, vestibulum ut diam. Maecenas vitae diam ligula.

Fusce ac pretium leo, in convallis augue. Mauris pulvinar elit rhoncus velit auctor finibus. Praesent et commodo est, eu luctus arcu. Vivamus ut porta tortor, eget facilisis ex. Nunc aliquet tristique mauris id sollicitudin. Donec quis commodo metus, sit amet accumsan nibh. Cum sociis natoque penatibus et magnis dis parturient montes, nascetur ridiculus mus.

Duis elementum dictum tristique. Integer mattis libero sit amet pretium euismod. Curabitur auctor eu augue ut ornare. Integer bibendum eros ullamcorper rhoncus convallis. Pellentesque non pretium ligula, sit amet bibendum eros. Nam venenatis ex felis, quis blandit nunc auctor sit amet. Maecenas ut eros pharetra, lobortis neque id, fermentum arcu. Cras neque dui, rhoncus feugiat leo id, semper facilisis lorem. Fusce non ex turpis. Nullam venenatis sed ligula ac lacinia.

\textbf{Palavras-chaves}: lorem. ipsum. dolor. sit. amet.
\end{resumo}

% Insere lista de ilustrações.
\pdfbookmark[0]{\listfigurename}{lof}
\listoffigures*
\cleardoublepage

% Insere lista de tabelas.
\pdfbookmark[0]{\listtablename}{lot}
\listoftables*
\cleardoublepage

% Lista de abreviaturas e siglas.
% (*) Preencher com as siglas usadas ao longo do texto e seus significados.
\begin{siglas}
  \item[UML] Unified Modeling Language
\end{siglas}

% Insere o sumário.
\pdfbookmark[0]{\contentsname}{toc}
\tableofcontents*
\cleardoublepage



%%% Início da parte de conteúdo do documento. %%%

% Marca o início dos elementos textuais.
\textual

% Inclusão dos capítulos.
% (*) Para facilitar a organização, os capítulos foram divididos em arquivo separados e colocados dentro da.
% pasta capitulos/. Caso o aluno prefira trabalhar com um só arquivo, basta substituir os comandos \include 
% pelos conteúdos dos arquivos que estão sendo incluídos, excluindo a pasta capitulos/ em seguida.
% ==============================================================================
% PG - Nome do Aluno
% Capítulo 1 - Introdução
% ==============================================================================
\chapter{Introdução}
\label{sec-intro}

O Capítulo de Introdução deve apresentar o contexto, motivação e justificativa do trabalho, seus objetivos, método de desenvolvimento e organização da monografia. Deve conter de 3 a 5 páginas.


%%% Início de seção. %%%
\section{Motivação e Justificativa}
\label{sec-intro-motjus}

A \textbf{Motivação} apresenta as circunstâncias que levaram à escolha do tema abordado e ao desenvolvimento do que é proposto no trabalho. A \textbf{Justificativa} apresenta o porquê da escolha do tema e do problema tratado e destaca a relevância do trabalho, referindo-se a estudos anteriores sobre o tema, ressaltando suas eventuais limitações e destacando a necessidade de se continuar explorando o assunto.


%%% Início de seção. %%%
\section{Objetivos}
\label{sec-intro-obj}

Nesta subseção, deve ser descrito o objetivo geral do trabalho, detalhando em seguida, seus objetivos específicos. O \textbf{Objetivo Geral} expressa a finalidade principal do trabalho: para quê? Deve ter coerência direta com o tema do trabalho e ser apresentado em uma frase que inicie com um verbo no infinitivo. O objetivo geral do trabalho está relacionado ao resultado principal do trabalho. Os \textbf{Objetivos Específicos} apresentam os detalhes ou desdobramentos do objetivo geral que levam a resultados intermediários e relevantes para alcançar o objetivo geral. Sempre será mais de um objetivo específico, todos iniciando com verbo no infinitivo.


%%% Início de seção. %%%
\section{Método de Desenvolvimento do Trabalho}
\label{sec-intro-met}

Nesta subseção, deve ser apresentado o \textbf{Método de Desenvolvimento} (ou o \textbf{Método de Pesquisa}, quando for o caso) utilizado no trabalho. Aqui são apresentadas as atividades realizadas e os procedimentos/técnicas que foram usados durante o desenvolvimento do trabalho.


%%% Início de seção. %%%
\section{Organização da Monografia}
\label{sec-intro-organizacao}

Por fim, a última subseção da monografia apresenta a estrutura do texto. Por exemplo, para este documento esta seção poderia conter o seguinte texto:

Além desta introdução, este modelo de monografia é composto por outros cinco capítulos:

\begin{itemize}
	\item O Capítulo~\ref{sec-referencial} apresenta os aspectos relativos ao conteúdo teórico relevante para o trabalho;
	
	\item O Capítulo~\ref{sec-contribuicao} apresenta a principal contribuição do trabalho;
	
	\item O Capítulo~\ref{sec-avaliacao} apresenta a avaliação da proposta, quando a mesma tiver sido realizada e requeira uma descrição detalhada;
	
	\item O Capítulo~\ref{sec-conclusoes} apresenta as considerações finais do trabalho;
	
	\item O Capítulo~\ref{sec-dicaslatex} traz dicas básicas para escrita de textos científicos em \latex.
\end{itemize}



% ==============================================================================
% TCC - Nome do Aluno
% Capítulo 2 - Referencial Teórico
% ==============================================================================
\chapter{Fundamentação Teórica e Tecnologias Utilizadas}
\label{sec-fundteo}

Neste capítulo serão apresentados os referenciais teóricos utilizados para o desenvolvimento
deste trabalho. A Seção \ref{sec-fundteo-engsoft} aborda os conceitos da Engenharia de Software,
em seguida, o método FrameWeb é explicitado na Seção \ref{sec-fundteo-frameweb}
e por fim, os \textit{frameworks} suportados pelo método, assim como o \textit{framework}
utilizado neste trabalho são apresentados na Seção \ref{sec-fundteo-framework}.

%%%%%%%%% Início de seção. %%%%%%%%%
\section{Engenharia de Software}
\label{sec-fundteo-engsoft}

A Engenharia de Software é uma área da computação que se preocupa com todo o ciclo de vida do software,
desde a especificação, desenvolvimento e até manutenção de sistemas de software \cite{sommerville:2011}.
Por meio da aplicação de métodos e ferramentas que possibilitam a construção de sistemas complexos,
tem como objetivo gerar produtos dentro do prazo e com qualidade \cite{pressman:2011}.
Segundo \citeonline{sommerville:2011}, a engenharia de software define quatro atividades
essenciais para todos os processos de software, são elas:


\begin{enumerate}
    \item \textbf{Especificação de software}: São definidas as funcionalidades do 
        sistema, a partir da comunicação cliente e engenheiro.
    \item \textbf{Projeto e implementação de software}: São definidos os modelos 
        arquiteturais e o sistema é implementado.
    \item \textbf{Validação de software}: O software deve ser validado para garantir 
        que os requisitos estejam contemplados.
    \item \textbf{Evolução de software}: São feitas adaptações e evoluções no sistema 
        para atender as necessidades do cliente.
\end{enumerate}

\subsection{Engenharia de Requisitos}
\label{subsec-fundteo-engsoft-engreq}

Como visto anteriormente, é na fase de Especificação de Software que as funcionalidades 
e restrições do sistema são definidas, e é fundamental que os softwares contemplem os
requisitos estabelecidos para garantir um desempenho satisfatório no suporte aos processos 
de negócios. Dessa forma, uma importante tarefa no desenvolvimento de software é a
identificação dos requisitos dos negócios que os sistemas vão apoiar \cite{falbo:2017}.
É neste contexto que entra a Engenharia de Requisitos, ação de engenharia de 
software que ocorre durante as atividades de comunicação e modelagem \cite{pressman:2011},
responsável por descobrir, analisar, documentar e verificar esses serviços e restrições \cite{sommerville:2011}. 

\citeonline{sommerville:2011} classifica requisitos em duas categorias, sendo elas:

\begin{itemize}
    \item \textbf{Requisitos funcionais}: Descrevem as funcionalidades que o sistema deve 
        fornecer, ou seja, como o sistema deve se comportar para determinadas entradas.
    \item \textbf{Requisitos não funcionais}: Descrevem restrições sobre os serviços ou
        funções oferecidas pelo sistema, como por exemplo, desempenho, confiabilidade e 
        disponibilidade. 
\end{itemize}

Além dos requisitos funcionais e não funcionais, é importante dar destaque às Regras de
Negócio, requisitos provenientes do domínio de aplicação do sistema que refletem características
e restrições do mesmo \cite{sommerville:2011,falbo:2014}.


\subsection{Engenharia Web}
\label{subsec-fundteo-engsoft-engweb}

Com o crescimento da Internet, foi significativo o impacto que a mesma causou em diversos
setores da economia, comércios utilizando sites para realizar suas vendas, indústrias com
sistemas para gerenciar seus processos e até mesmo em nossas vidas pessoais \cite{murugesan:2001}.
A rápida necessidade de sistemas complexos deixa de lado a preocupação por qualidade à longo prazo,
surgindo a chamada "Crise Web" \cite{murugesan:2001}, uma variação potencialmente mais séria da 
conhecida "Crise de Software" \cite{gibbs:1994}. É importante ressaltar que, embora os sistemas Web
sejam softwares, eles possuem características e requisitos exclusivos \cite{pressman:2011},
o que trouxe a necessidade de uma nova área da Engenharia de Software, a Engenharia Web.

A Engenharia Web pode então ser descrita como a aplicação de princípios e métodos da Engenharia
de Software, adaptados ao desenvolvimento de sistemas Web e suas características específicas 
\cite{beder:2017,murugesan:2001}, com o objetivo de garantir a qualidade dos sistemas Web. 
\cite{olsina:2001} define um conjunto de atributos técnicos que levam a qualidade de um sistema Web,
são eles:

\begin{itemize}
    \item \textbf{Usabilidade}: O sistema deve ser fácil de usar, com uma interface intuitiva e 
        que atenda as necessidades do usuário.
    \item \textbf{Funcionabilidade}: O sistema deve funcionar corretamente, atendendo às características
        do domínio.
    \item \textbf{Eficiência}: O sistema deve fornecer respostas rápidas e precisas.
    \item \textbf{Confiabilidade}: O sistema deve ser confiável, e capaz de se recuperar de erros.
    \item \textbf{Manutenibilidade}: O sistema deve ser fácil de ser corrigido, adaptado e melhorado.
\end{itemize}

A Engenharia Web se baseia em dois conceitos para atingir a qualidade dos sistemas Web: Agilidade e Arcabouço de Processo.
A agilidade é uma abordagem de desenvolvimento de software que se baseia em ciclos curtos de desenvolvimento,
já o Arcabouço de Processo é um conjunto de atividades que devem ser realizadas ao longo de todo o processo
de desenvolvimento do sistema, independente do seu tamanho e complexidade \cite{beder:2017}.


%%%%%%%%% Início de seção. %%%%%%%%%
\section{FrameWeb}
\label{sec-fundteo-frameweb}



%%%%%%%%% Início de seção. %%%%%%%%%
\section{Frameworks}
\label{sec-fundteo-framework}


% ==============================================================================
% PG - Nome do Aluno
% Capítulo 3 - Contribuição
% ==============================================================================
\chapter{Contribuição do Trabalho}
\label{sec-contribuicao}

Este capítulo deve apresentar a principal contribuição do trabalho. Caso o aluno e orientador desejem, o título do capítulo pode ser alterado para referenciar diretamente a contribuição (por exemplo, PIS: Plataforma para Integração de Serviços; Um Sistema para Controle de Processos da UFES, Solução de Otimização para Carregamento de Contêineres; etc.)

O capítulo deve ser estruturado em seções de forma a apresentar de forma clara e com todas as informações necessárias, a contribuição do trabalho. Por exemplo, caso a contribuição produzida seja um sistema de informação, espera-se que sejam apresentados seus requisitos, funcionalidades, modelos (p.ex., modelo estrutural, modelo da arquitetura, etc.) e telas do sistema. Caso seja uma plataforma, espera-se que a plataforma como um todo seja apresentada e que seus componentes sejam descritos sejam apropriadamente.


% ==============================================================================
% PG - Nome do Aluno
% Capítulo 4 - Avaliação
% ==============================================================================
\chapter{Avaliação da Proposta}
\label{sec-avaliacao}

Este capítulo deve ser incluso na monografia quando tiver sido realizado algum tipo de avaliação da proposta que requeira uma descrição detalhada (por exemplo, experimentos, simulações, etc.) O capítulo deve apresentar a avaliação realizada, deixando claro qual foi objetivo da avaliação, os passos realizados, os resultados obtidos e a interpretação desses resultados considerando o objetivo inicial. Em casos em que a avaliação realizada não demande um capítulo dedicado a ela (por ser muito simples ou pequena, por exemplo), ela pode ser tratada em uma seção específica no capítulo anterior.

% ==============================================================================
% PG - Nome do Aluno
% Capítulo 5 - Considerações Finais
% ==============================================================================
\chapter{Conclusão}
\label{sec-conclusoes}

% \hl{Neste capítulo devem ser realizadas as considerações finais do trabalho, sendo apresentadas suas principais contribuições, limitações, lições aprendidas durante o desenvolvimento do trabalho, dificuldades enfrentadas e perspectivas de trabalhos futuros. O capítulo deve ter entre 3 e 5 páginas.}

Este capítulo apresenta as considerações finais do trabalho, dificuldades, e perspectivas para trabalhos futuros.
Na Seção~\ref{sec-conclusoes-consideracoes}, são apresentados as conclusões obtidas e suas relações com os objetivos definidos, enquanto
na Seção~\ref{sec-conclusoes-trabalhosfuturos} são apresentadas ideias e melhorias para trabalhos futuros.

%%% Início de seção. %%%
\section{Considerações Finais}
\label{sec-conclusoes-consideracoes}

% \hl{Esta seção deve apresentar um texto de fechamento do trabalho, devendo incluir considerações sobre o trabalho desenvolvido, suas limitações, contribuições, experiência adquirida pelo aluno e lições aprendidas ao longo do desenvolvimento, bem como dificuldades enfrentadas durante o desenvolvimento do trabalho. Nesta seção é preciso mostrar claramente a relação entre os resultados produzidos no trabalho e os objetivos estabelecidos no Capítulo}

% Objetivos
% • Compreender o método FrameWeb;
% • Analisar os requisitos da aplicação SCAP;
% • Implementar a aplicação SCAP utilizando o framework Next.js e o método FrameWeb

Neste projeto uma nova implementação do SCAP aplicando o método FrameWeb foi feita, dessa vez utilizando o \textit{framework} Next.js.
Os modelos FrameWeb produzidos foram essenciais para a implementação do SCAP, pois forneceram uma visão geral da aplicação, facilitando e agilizando o desenvolvimento do sistema.
Mesmo assim, ao longo da implementação os modelos foram revisitados diversas vezes, e inconsistências tiveram de ser consertadas.

Portanto, revisitando os objetivos do projeto, foi possível estudar e compreender o método FrameWeb para aplicá-lo na implementação do SCAP, considerando os requisitos levantados anteriormente.
Com isso, as disciplinas de Engenharia de Software foram revisitadas e colocadas em prática, e o conhecimento adquirido foi essencial para o andamento deste projeto.
Além disso, o projeto permitiu o aprendizado de um novo e moderno \textit{framework} de desenvolvimento, o Next.js.

% Limitações
Uma limitação do método FrameWeb é o fato de este ter sido construído em cima de projetos Java, que não refletem a realidade do desenvolvimento de aplicações web modernas
SPA. Isso tornou a implementação mais complexa do que o esperado, uma vez que o Next.js não costuma seguir padrões como o \textit{DAO} ou ser Orientado a Objetos.
Tendo em vista que o foco do projeto era a aplicação e análise do método FrameWeb, duas funcionalidades ficaram não foram implementadas,
sendo elas: o envio de \textit{e-mails} e a geração de atas. Ambas funcionalidades são complexas e demandariam um tempo maior para serem implementadas.

Apesar do método FrameWeb utilizar UML, o que facilitaria a modelagem dos diagramas, a ferramenta utilizada para a modelagem, o \textit{Visual Paradigm}, 
tem uma curva de aprendizado acentuada, o que inicialmente dificultou a modelagem dos diagramas. Além disso, a ferramenta ainda não possui suporte para a geração de código
a partir dos diagramas, portanto o código foi implementado por inteiro do zero.


%%% Início de seção. %%%
\section{Trabalhos Futuros}
\label{sec-conclusoes-trabalhosfuturos}

% \hl{Nesta seção devem ser identificados trabalhos futuros que poderão ser realizados a partir dos resultados obtidos até o momento no trabalho. Idealmente, trabalhos futuros não devem apenas ser citados. Recomenda-se discutir aspectos sobre como podem ser realizados e por que é importante que sejam realizados (que benefícios podem ser obtidos com sua realização).}

Para trabalhos futuros, seria interessante analisar se os requisitos de fato contemplam as necessidades do departamento,
implementar as funcionalidades que ficaram pendentes, e fazer o \textit{deploy} da aplicação em um servidor.

Além disso, utilizar as ferramentas de modelagem, sejam elas o Editor FrameWeb~\cite{campos:2017} ou o \textit{Visual Paradigm} com o plugin desenvolvido~\cite{silva:2023},
com aplicações utilizando JavaScript, assim como testar o gerador de código.

A grande maioria das aplicações, dos trabalhos anteriores, foram feitas utilizando \textit{frameworks} Java ou JavaScript, e seria interessante
testar também a implementação com \textit{frameworks} Python,\footnote{Python, \url{https://www.python.org}} como Django, ou com \textit{frameworks} Ruby,\footnote{Ruby, \url{https://www.ruby-lang.org/en/}} como Ruby on Rails.
Por fim, para deixar o modelo mais robusto, é importante comparar todos os resultados obtidos ao longo de todas as diferentes implementações feitas.

\include{capitulos/ch6-dicaslatex}



%%% Páginas finais do documento: bibliografia e anexos. %%%

% Finaliza a parte no bookmark do PDF para que se inicie o bookmark na raiz e adiciona espaço de parte no sumário.
\phantompart

% Marca o início dos elementos pós-textuais.
\postextual

% Referências bibliográficas
\bibliography{bibliografia}


% Apêndices.
\begin{apendicesenv}

% Imprime uma página indicando o início dos apêndices.
\partapendices

% (*) Incluir como apêndice a documentação técnica produzida durante o PG (especificação de requisitos,
% projeto arquitetural, etc.). Utilizar o exemplo \includepdf caso o documento seja produzido em outro
% editor de texto (Microsoft Word, LibreOffice Writer) e transformado em PDF. Utilizar o exemplo \include
% caso os documentos tenham sido também escritos em LaTeX.
% \includepdf[pages={1-}]{apendices/apendice01-requisitos.pdf}
% \includepdf[pages={1-}]{apendices/apendice02-projeto.pdf}
% \include{ap1-requisitos}
% \include{ap2-projeto}
\end{apendicesenv}


% Índice remissivo.
\phantompart
\printindex

% Fim do documento.
\end{document}
