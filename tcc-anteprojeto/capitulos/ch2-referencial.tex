% ==============================================================================
% TCC - Nome do Aluno
% Capítulo 2 - Referencial Teórico
% ==============================================================================
\chapter{Fundamentação Teórica e Tecnologias Utilizadas}
\label{sec-fundteo}

Neste capítulo serão apresentados os referenciais teóricos utilizados para o desenvolvimento
deste trabalho. A Seção \ref{sec-fundteo-engsoft} aborda os conceitos da Engenharia de Software,
em seguida, o método FrameWeb é explicitado na Seção \ref{sec-fundteo-frameweb}
e por fim, os \textit{frameworks} suportados pelo método, assim como o \textit{framework}
utilizado neste trabalho são apresentados na Seção \ref{sec-fundteo-framework}.

%%%%%%%%% Início de seção. %%%%%%%%%
\section{Engenharia de Software}
\label{sec-fundteo-engsoft}

A Engenharia de Software é uma área da computação que se preocupa com todo o ciclo de vida do software,
desde a especificação, desenvolvimento e até manutenção de sistemas de software \cite{sommerville:2011}.
Por meio da aplicação de métodos e ferramentas que possibilitam a construção de sistemas complexos,
tem como objetivo gerar produtos dentro do prazo e com qualidade \cite{pressman:2011}.
Segundo \citeonline{sommerville:2011}, a engenharia de software define quatro atividades
essenciais para todos os processos de software, são elas:


\begin{enumerate}
    \item \textbf{Especificação de software}: São definidas as funcionalidades do 
        sistema, a partir da comunicação cliente e engenheiro.
    \item \textbf{Projeto e implementação de software}: São definidos os modelos 
        arquiteturais e o sistema é implementado.
    \item \textbf{Validação de software}: O software deve ser validado para garantir 
        que os requisitos estejam contemplados.
    \item \textbf{Evolução de software}: São feitas adaptações e evoluções no sistema 
        para atender as necessidades do cliente.
\end{enumerate}

\subsection{Engenharia de Requisitos}
\label{subsec-fundteo-engsoft-engreq}

Como visto anteriormente, é na fase de Especificação de Software que as funcionalidades 
e restrições do sistema são definidas, e é fundamental que os softwares contemplem os
requisitos estabelecidos para garantir um desempenho satisfatório no suporte aos processos 
de negócios. Dessa forma, uma importante tarefa no desenvolvimento de software é a
identificação dos requisitos dos negócios que os sistemas vão apoiar \cite{falbo:2017}.
É neste contexto que entra a Engenharia de Requisitos, ação de engenharia de 
software que ocorre durante as atividades de comunicação e modelagem \cite{pressman:2011},
responsável por descobrir, analisar, documentar e verificar esses serviços e restrições \cite{sommerville:2011}. 

\citeonline{sommerville:2011} classifica requisitos em duas categorias, sendo elas:

\begin{itemize}
    \item \textbf{Requisitos funcionais}: Descrevem as funcionalidades que o sistema deve 
        fornecer, ou seja, como o sistema deve se comportar para determinadas entradas.
    \item \textbf{Requisitos não funcionais}: Descrevem restrições sobre os serviços ou
        funções oferecidas pelo sistema, como por exemplo, desempenho, confiabilidade e 
        disponibilidade. 
\end{itemize}

Além dos requisitos funcionais e não funcionais, é importante dar destaque às Regras de
Negócio, requisitos provenientes do domínio de aplicação do sistema que refletem características
e restrições do mesmo \cite{sommerville:2011,falbo:2014}.


\subsection{Engenharia Web}
\label{subsec-fundteo-engsoft-engweb}

Com o crescimento da Internet, foi significativo o impacto que a mesma causou em diversos
setores da economia, comércios utilizando sites para realizar suas vendas, indústrias com
sistemas para gerenciar seus processos e até mesmo em nossas vidas pessoais \cite{murugesan:2001}.
A rápida necessidade de sistemas complexos deixa de lado a preocupação por qualidade à longo prazo,
surgindo a chamada "Crise Web" \cite{murugesan:2001}, uma variação potencialmente mais séria da 
conhecida "Crise de Software" \cite{gibbs:1994}. É importante ressaltar que, embora os sistemas Web
sejam softwares, eles possuem características e requisitos exclusivos \cite{pressman:2011},
o que trouxe a necessidade de uma nova área da Engenharia de Software, a Engenharia Web.

A Engenharia Web pode então ser descrita como a aplicação de princípios e métodos da Engenharia
de Software, adaptados ao desenvolvimento de sistemas Web e suas características específicas 
\cite{beder:2017,murugesan:2001}, com o objetivo de garantir a qualidade dos sistemas Web. 
\cite{olsina:2001} define um conjunto de atributos técnicos que levam a qualidade de um sistema Web,
são eles:

\begin{itemize}
    \item \textbf{Usabilidade}: O sistema deve ser fácil de usar, com uma interface intuitiva e 
        que atenda as necessidades do usuário.
    \item \textbf{Funcionabilidade}: O sistema deve funcionar corretamente, atendendo às características
        do domínio.
    \item \textbf{Eficiência}: O sistema deve fornecer respostas rápidas e precisas.
    \item \textbf{Confiabilidade}: O sistema deve ser confiável, e capaz de se recuperar de erros.
    \item \textbf{Manutenibilidade}: O sistema deve ser fácil de ser corrigido, adaptado e melhorado.
\end{itemize}

A Engenharia Web se baseia em dois conceitos para atingir a qualidade dos sistemas Web: Agilidade e Arcabouço de Processo.
A agilidade é uma abordagem de desenvolvimento de software que se baseia em ciclos curtos de desenvolvimento,
já o Arcabouço de Processo é um conjunto de atividades que devem ser realizadas ao longo de todo o processo
de desenvolvimento do sistema, independente do seu tamanho e complexidade \cite{beder:2017}.


%%%%%%%%% Início de seção. %%%%%%%%%
\section{FrameWeb}
\label{sec-fundteo-frameweb}



%%%%%%%%% Início de seção. %%%%%%%%%
\section{Frameworks}
\label{sec-fundteo-framework}

