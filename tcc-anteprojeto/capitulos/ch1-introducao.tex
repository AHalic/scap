% ==============================================================================
% TCC - Nome do Aluno
% Capítulo 1 - Introdução
% ==============================================================================
\chapter{Introdução}
\label{sec-intro}

%\hl{Texto.}

%\hrulefill

A \textbf{Introdução} deve conter de \textbf{3 a 5 páginas}. Primeiramente, deve ser colocada a
Descrição do trabalho, a qual apresenta o contexto do trabalho e a definição do escopo do
mesmo. Deve-se delimitar o escopo do trabalho de forma que haja condições
técnicas suficientes para que o mesmo seja concluído em tempo hábil.


%%% Início de seção. %%%
\section{Motivação e Justificativa}
\label{sec-intro-motjus}

A \textbf{Motivação} apresenta as circunstancias que interferiram na escolha do tema.
A \textbf{Justificativa} apresenta o porquê da escolha do tema, o problema a ser resolvido
e a relevância do trabalho, referindo-se a estudos anteriores sobre o tema, ressaltando
suas eventuais limitações e destacando a necessidade de se continuar pesquisando o
assunto.


%%% Início de seção. %%%
\section{Objetivos}
\label{sec-intro-obj}

Nesta subseção, deve ser descrito o \textbf{objetivo geral} do trabalho, detalhando em
seguida, seus \textbf{objetivos específicos}.

O \textbf{Objetivo Geral} expressa a finalidade do trabalho: para quê? Deve ter coerência
direta com o tema do trabalho e ser apresentado em uma frase que inicie com um verbo
no infinitivo. O objetivo geral do trabalho está relacionado ao resultado principal do trabalho.

Os \textbf{Objetivos Específicos} apresentam os detalhes e/ou desdobramentos do
objetivo geral que levam a resultados intermediários e relevantes para alcançar o objetivo geral. Sempre será mais de um objetivo específico, todos iniciando com verbo no infinitivo.


%OsObjetivos Específicosapresentam os detalhes e/ou desdobramento do objetivogeral. Sempre serão mais de um objetivo, todos iniciando com verbo no infinitivo queapresente tarefas parciais de pesquisa em prol da execução do objetivo geral.


%%% Início de seção. %%%
\section{Método de Desenvolvimento do Trabalho}
\label{sec-intro-met}

Nesta subseção, deve ser apresentado o \textbf{Método de Desenvolvimento} (ou o \textbf{Método de Pesquisa}, quando for o caso) do trabalho. Aqui são apresentados os procedimentos/técnicas que serão usados durante o desenvolvimento do trabalho. 

%%% Início de seção. %%%
\section{Cronograma}
\label{sec-intro-crono}

O \textbf{Cronograma de Execução} apresenta a distribuição no tempo das atividades
que deverão ser desenvolvidas ao longo do trabalho. As atividades apresentadas no cronograma devem estar alinhadas com os objetivos apresentados na Subseção~\ref{sec-intro-obj}, ou seja, elas devem ser capazes de produzir os resultados necessários para alcançar os objetivos estabelecidos. Deve-se apresentar uma listagem com a descrição de cada uma dessas atividades e em seguida mostrada uma tabela contendo todas as atividades previstas juntamente com a  previsão do período de execução de cada uma.

Um exemplo seria:
\begin{itemize}
\item Atividade 1: detalhamento ...
\item Atividade 2: detalhamento ...
\item ...
\end{itemize}

\begin{table}[h]
\centering
\begin{tabular}{c|c|c|c|c|c|c|}
\cline{2-7}
\multicolumn{1}{l|}{} & \textbf{(mês)} & \textbf{(mês)} & \textbf{(mês)} & \textbf{(mês)} & \textbf{(mês)} & \textbf{(mês)} \\ \hline
\multicolumn{1}{|c|}{\textbf{Atividade 1}} & X & X &  &  &  &  \\ \hline
\multicolumn{1}{|c|}{\textbf{Atividade 2}} &  & X & X &  &  &  \\ \hline
\multicolumn{1}{|c|}{\textbf{Atividade 3}} &  &  &  & X &  &  \\ \hline
\multicolumn{1}{|c|}{\textbf{Atividade 4}} &  & X &  & X &  &  \\ \hline
\multicolumn{1}{|c|}{\textbf{. . .}} &  &  &  &  & X & X \\ \hline
\end{tabular}
\end{table}