% ==============================================================================
% Modelo para Anteprojeto de Graduação - Ciência da Computação (em Português)
% Prof. Vítor E. Silva Souza - NEMO/UFES :: DI/UFES :: PPGI/UFES
% Com alterações feitas pelo Colegiado de Engenharia de Computação / CT / UFES
%
% Baseado em abtex2-modelo-trabalho-academico.tex, v-1.9.2 laurocesar
% Copyright 2012-2014 by abnTeX2 group at http://abntex2.googlecode.com/ 
%
% This work may be distributed and/or modified under the conditions of the LaTeX 
% Project Public License, either version 1.3 of this license or (at your option) 
% any later version. The latest version of this license is in
% http://www.latex-project.org/lppl.txt.
%
% IMPORTANTE:
% Instruções encontram-se espalhadas pelo documento. Para facilitar sua leitura,
% tais instruções são precedidas por (*) -- utilize a função localizar do seu
% editor para passar por todas elas.
% ==============================================================================

% Usa o estilo abntex2, configurando detalhes de formatação e hifenização.
\documentclass[
	12pt,				% Tamanho da fonte.
	openright,			% Capítulos começam em página ímpar (insere página vazia caso preciso).
	oneside,			% Para impressão em verso e anverso. Oposto a oneside.
	a4paper,			% Tamanho do papel.
	english,			% Idioma adicional para hifenização.
	french,				% Idioma adicional para hifenização.
	spanish,			% Idioma adicional para hifenização.
	brazil				% O último idioma é o principal do documento.
	]{abntex2}



%%% Importação de pacotes. %%%

% Conserta o erro "No room for a new \count"
\usepackage{etex}
% Em alguns sistemas operacionais, a linha \reserveinserts{28} pode ser necessária.
%\reserveinserts{28}

% Usa a fonte Latin Modern.
\usepackage{lmodern}

% Seleção de códigos de fonte.
\usepackage[T1]{fontenc}

% Codificação do documento em Unicode.
\usepackage[utf8]{inputenc}

% Usado pela ficha catalográfica.
\usepackage{lastpage}

% Indenta o primeiro parágrafo de cada seção.
\usepackage{indentfirst}

% Controle das cores.
\usepackage[usenames,dvipsnames]{xcolor}

% Inclusão de gráficos.
\usepackage{graphicx}

% Posicionamento de elementos.
\usepackage{float}

% Melhor controle de layout em tabelas.
\usepackage{tabularx}
\usepackage{multirow}
\usepackage{hhline}

% Inclusão de páginas em PDF diretamente no documento (para uso nos apêndices).
\usepackage{pdfpages}

% Para melhorias de justificação.
\usepackage{microtype}

% Citações padrão ABNT.
\usepackage[brazilian,hyperpageref]{backref}
\usepackage[alf]{abntex2cite}	
\renewcommand{\backrefpagesname}{Citado na(s) página(s):~}		% Usado sem a opção hyperpageref de backref.
\renewcommand{\backref}{}										% Texto padrão antes do número das páginas.
\renewcommand*{\backrefalt}[4]{									% Define os textos da citação.
	\ifcase #1
		Nenhuma citação no texto.
	\or
		Citado na página #2.
	\else
		Citado #1 vezes nas páginas #2.
	\fi}

% \rm is deprecated and should not be used in a LaTeX2e document
% http://tex.stackexchange.com/questions/151897/always-textrm-never-rm-a-counterexample
\renewcommand{\rm}{\textrm}

% Pacotes não incluídos no template abntex2. 
% Podem ser comentados caso não queira utilizá-los.

% Inclusão de símbolos não padrão.
\usepackage{amssymb}
\usepackage{eurosym}

% Para utilizar \eqref para referenciar equações.
\usepackage{amsmath}

% Permite mostrar figuras muito largas em modo paisagem com \begin{sidewaysfigure} ao invés de \begin{figure}.
\usepackage{rotating}

% Permite customizar listas enumeradas/com marcadores.
\usepackage{enumitem}

% Permite inserir hiperlinks com \url{}.
\usepackage{bigfoot}
\usepackage{hyperref}

% Permite usar o comando \hl{} para evidenciar texto com fundo amarelo. Útil para chamar atenção a itens a fazer.
\usepackage{soulutf8}

% Permite inserir espaço em branco condicional (incluído no texto final só se necessário) em macros.
\usepackage{xspace}

% Permite inserir comentários para controle de revisão de documentos.
\usepackage[colorinlistoftodos, textwidth=20mm, textsize=footnotesize]{todonotes}
\newcommand{\aluno}[1]{\todo[author=\textbf{Aluno},color=green!30,caption={},inline]{#1}}
\newcommand{\professor}[1]{\todo[author=\textbf{Professor},color=red!30,caption={},inline]{#1}}

% Permite inserir espaço em branco condicional (incluído no texto final só se necessário) em macros.
\usepackage{xspace}

% Permite incluir listagens de código com o comando \lstinputlisting{}.
\usepackage{listings}
\usepackage{caption}
\DeclareCaptionFont{white}{\color{white}}
\DeclareCaptionFormat{listing}{\colorbox{gray}{\parbox{\textwidth}{#1#2#3}}}
\captionsetup[lstlisting]{format=listing,labelfont=white,textfont=white}
\renewcommand{\lstlistingname}{Listagem}
\definecolor{mygray}{rgb}{0.5,0.5,0.5}
\lstset{
	basicstyle=\scriptsize,
	breaklines=true,
	numbers=left,
	numbersep=5pt,
	numberstyle=\tiny\color{mygray}, 
	rulecolor=\color{black},
	showstringspaces=false,
	tabsize=2,
    inputencoding=utf8,
    extendedchars=true,
    literate=%
    {é}{{\'{e}}}1
    {è}{{\`{e}}}1
    {ê}{{\^{e}}}1
    {ë}{{\¨{e}}}1
    {É}{{\'{E}}}1
    {Ê}{{\^{E}}}1
    {û}{{\^{u}}}1
    {ù}{{\`{u}}}1
    {â}{{\^{a}}}1
    {à}{{\`{a}}}1
    {á}{{\'{a}}}1
    {ã}{{\~{a}}}1
    {Á}{{\'{A}}}1
    {Â}{{\^{A}}}1
    {Ã}{{\~{A}}}1
    {ç}{{\c{c}}}1
    {Ç}{{\c{C}}}1
    {õ}{{\~{o}}}1
    {ó}{{\'{o}}}1
    {ô}{{\^{o}}}1
    {Õ}{{\~{O}}}1
    {Ó}{{\'{O}}}1
    {Ô}{{\^{O}}}1
    {î}{{\^{i}}}1
    {Î}{{\^{I}}}1
    {í}{{\'{i}}}1
    {Í}{{\~{Í}}}1
}



%%% Definição de variáveis. %%%

% (*) Substituir os textos abaixo com as informações apropriadas.
\titulo{Título do Trabalho}
\autor{Nome do Aluno}
\local{Vitória, ES}
\data{\the\year}
\orientador{Prof. Dr. Fulano de Tal}
\coorientador{Prof. Dr. Ciclano de Tal}
\instituicao{
  Universidade Federal do Espírito Santo -- UFES
  \par
  Centro Tecnológico
  \par
  Colegiado do Curso de Ciência da Computação}
\tipotrabalho{Monografia (PG)}

% Preâmbulo (tipo do trabalho, objetivo, nome da instituição, área de concentração, etc.).
% (*) Verificar se está correto (ex.: substituir por Engenharia de Computação se for o caso).
\preambulo{Anteprojeto apresentado ao Colegiado do Curso de Ciência da Computação do Centro Tecnológico da Universidade Federal do Espírito Santo, como requisito para aprovação na Disciplina Projeto de Graduação I.}

% Macros específicas do trabalho.
% (*) Inclua aqui termos que são utilizados muitas vezes e que demandam formatação especial.
% Os exemplos abaixo incluem i* (substituindo o asterisco por uma estrela) e Java com TM em superscript.
% Use sempre \xspace para que o LaTeX inclua espaço em branco após a macro somente quando necessário.
\newcommand{\istar}{\textit{i}$^\star$\xspace}
\newcommand{\java}{Java\texttrademark\xspace}
\newcommand{\latex}{\LaTeX\xspace}




%%% Configurações finais de aparência. %%%

% Altera o aspecto da cor azul.
\definecolor{blue}{RGB}{41,5,195}

% Informações do PDF.
\makeatletter
\hypersetup{
	pdftitle={\@title}, 
	pdfauthor={\@author},
	pdfsubject={\imprimirpreambulo},
	pdfcreator={LaTeX with abnTeX2},
	pdfkeywords={abnt}{latex}{abntex}{abntex2}{trabalho acadêmico}, 
	colorlinks=true,				% Colore os links (ao invés de usar caixas).
	linkcolor=blue,					% Cor dos links.
	citecolor=blue,					% Cor dos links na bibliografia.
	filecolor=magenta,				% Cor dos links de arquivo.
	urlcolor=blue,					% Cor das URLs.
	bookmarksdepth=4
}
\makeatother

% Espaçamentos entre linhas e parágrafos.
\setlength{\parindent}{1.3cm}
\setlength{\parskip}{0.2cm}



%%% Páginas iniciais do documento: capa, folha de rosto, ficha, resumo, tabelas, etc. %%%

% Compila o índice.
\makeindex

% Inicia o documento.
\begin{document}

% Brasão da instituição.
\begin{figure}
\centering
\includegraphics[width=.20\textwidth]{figuras/brasao.jpg} 
\label{fig-brasao}
\end{figure}

\begin{center}
\textbf{\textsf{UNIVERSIDADE FEDERAL DO ESPÍRITO SANTO}}

\textbf{\textsf{CENTRO TECNOLÓGICO}}

\textbf{\textsf{COLEGIADO DO CURSO DE CIÊNCIA DA COMPUTAÇÃO}}

\large{\textbf{\textsf{  }}}

\large{\textbf{\textsf{  }}}
\end{center}

% Retira espaço extra obsoleto entre as frases.
\frenchspacing

% Capa do trabalho.
\imprimircapa

% Folha de rosto (o * indica que haverá a ficha bibliográfica).
\imprimirfolhaderosto*





%%% Início da parte de conteúdo do documento. %%%

% Marca o início dos elementos textuais.
\textual

% Inclusão dos capítulos.
% (*) Para facilitar a organização, os capítulos foram divididos em arquivo separados e colocados dentro da.
% pasta capitulos/. Caso o aluno prefira trabalhar com um só arquivo, basta substituir os comandos \include 
% pelos conteúdos dos arquivos que estão sendo incluídos, excluindo a pasta capitulos/ em seguida.
% ==============================================================================
% PG - Nome do Aluno
% Capítulo 1 - Introdução
% ==============================================================================
\chapter{Introdução}
\label{sec-intro}

O Capítulo de Introdução deve apresentar o contexto, motivação e justificativa do trabalho, seus objetivos, método de desenvolvimento e organização da monografia. Deve conter de 3 a 5 páginas.


%%% Início de seção. %%%
\section{Motivação e Justificativa}
\label{sec-intro-motjus}

A \textbf{Motivação} apresenta as circunstâncias que levaram à escolha do tema abordado e ao desenvolvimento do que é proposto no trabalho. A \textbf{Justificativa} apresenta o porquê da escolha do tema e do problema tratado e destaca a relevância do trabalho, referindo-se a estudos anteriores sobre o tema, ressaltando suas eventuais limitações e destacando a necessidade de se continuar explorando o assunto.


%%% Início de seção. %%%
\section{Objetivos}
\label{sec-intro-obj}

Nesta subseção, deve ser descrito o objetivo geral do trabalho, detalhando em seguida, seus objetivos específicos. O \textbf{Objetivo Geral} expressa a finalidade principal do trabalho: para quê? Deve ter coerência direta com o tema do trabalho e ser apresentado em uma frase que inicie com um verbo no infinitivo. O objetivo geral do trabalho está relacionado ao resultado principal do trabalho. Os \textbf{Objetivos Específicos} apresentam os detalhes ou desdobramentos do objetivo geral que levam a resultados intermediários e relevantes para alcançar o objetivo geral. Sempre será mais de um objetivo específico, todos iniciando com verbo no infinitivo.


%%% Início de seção. %%%
\section{Método de Desenvolvimento do Trabalho}
\label{sec-intro-met}

Nesta subseção, deve ser apresentado o \textbf{Método de Desenvolvimento} (ou o \textbf{Método de Pesquisa}, quando for o caso) utilizado no trabalho. Aqui são apresentadas as atividades realizadas e os procedimentos/técnicas que foram usados durante o desenvolvimento do trabalho.


%%% Início de seção. %%%
\section{Organização da Monografia}
\label{sec-intro-organizacao}

Por fim, a última subseção da monografia apresenta a estrutura do texto. Por exemplo, para este documento esta seção poderia conter o seguinte texto:

Além desta introdução, este modelo de monografia é composto por outros cinco capítulos:

\begin{itemize}
	\item O Capítulo~\ref{sec-referencial} apresenta os aspectos relativos ao conteúdo teórico relevante para o trabalho;
	
	\item O Capítulo~\ref{sec-contribuicao} apresenta a principal contribuição do trabalho;
	
	\item O Capítulo~\ref{sec-avaliacao} apresenta a avaliação da proposta, quando a mesma tiver sido realizada e requeira uma descrição detalhada;
	
	\item O Capítulo~\ref{sec-conclusoes} apresenta as considerações finais do trabalho;
	
	\item O Capítulo~\ref{sec-dicaslatex} traz dicas básicas para escrita de textos científicos em \latex.
\end{itemize}



% ==============================================================================
% TCC - Nome do Aluno
% Capítulo 2 - Referencial Teórico
% ==============================================================================
\chapter{Fundamentação Teórica e Tecnologias Utilizadas}
\label{sec-fundteo}

Neste capítulo serão apresentados os referenciais teóricos utilizados para o desenvolvimento
deste trabalho. A Seção \ref{sec-fundteo-engsoft} aborda os conceitos da Engenharia de Software,
em seguida, o método FrameWeb é explicitado na Seção \ref{sec-fundteo-frameweb}
e por fim, os \textit{frameworks} suportados pelo método, assim como o \textit{framework}
utilizado neste trabalho são apresentados na Seção \ref{sec-fundteo-framework}.

%%%%%%%%% Início de seção. %%%%%%%%%
\section{Engenharia de Software}
\label{sec-fundteo-engsoft}

A Engenharia de Software é uma área da computação que se preocupa com todo o ciclo de vida do software,
desde a especificação, desenvolvimento e até manutenção de sistemas de software \cite{sommerville:2011}.
Por meio da aplicação de métodos e ferramentas que possibilitam a construção de sistemas complexos,
tem como objetivo gerar produtos dentro do prazo e com qualidade \cite{pressman:2011}.
Segundo \citeonline{sommerville:2011}, a engenharia de software define quatro atividades
essenciais para todos os processos de software, são elas:


\begin{enumerate}
    \item \textbf{Especificação de software}: São definidas as funcionalidades do 
        sistema, a partir da comunicação cliente e engenheiro.
    \item \textbf{Projeto e implementação de software}: São definidos os modelos 
        arquiteturais e o sistema é implementado.
    \item \textbf{Validação de software}: O software deve ser validado para garantir 
        que os requisitos estejam contemplados.
    \item \textbf{Evolução de software}: São feitas adaptações e evoluções no sistema 
        para atender as necessidades do cliente.
\end{enumerate}

\subsection{Engenharia de Requisitos}
\label{subsec-fundteo-engsoft-engreq}

Como visto anteriormente, é na fase de Especificação de Software que as funcionalidades 
e restrições do sistema são definidas, e é fundamental que os softwares contemplem os
requisitos estabelecidos para garantir um desempenho satisfatório no suporte aos processos 
de negócios. Dessa forma, uma importante tarefa no desenvolvimento de software é a
identificação dos requisitos dos negócios que os sistemas vão apoiar \cite{falbo:2017}.
É neste contexto que entra a Engenharia de Requisitos, ação de engenharia de 
software que ocorre durante as atividades de comunicação e modelagem \cite{pressman:2011},
responsável por descobrir, analisar, documentar e verificar esses serviços e restrições \cite{sommerville:2011}. 

\citeonline{sommerville:2011} classifica requisitos em duas categorias, sendo elas:

\begin{itemize}
    \item \textbf{Requisitos funcionais}: Descrevem as funcionalidades que o sistema deve 
        fornecer, ou seja, como o sistema deve se comportar para determinadas entradas.
    \item \textbf{Requisitos não funcionais}: Descrevem restrições sobre os serviços ou
        funções oferecidas pelo sistema, como por exemplo, desempenho, confiabilidade e 
        disponibilidade. 
\end{itemize}

Além dos requisitos funcionais e não funcionais, é importante dar destaque às Regras de
Negócio, requisitos provenientes do domínio de aplicação do sistema que refletem características
e restrições do mesmo \cite{sommerville:2011,falbo:2014}.


\subsection{Engenharia Web}
\label{subsec-fundteo-engsoft-engweb}

Com o crescimento da Internet, foi significativo o impacto que a mesma causou em diversos
setores da economia, comércios utilizando sites para realizar suas vendas, indústrias com
sistemas para gerenciar seus processos e até mesmo em nossas vidas pessoais \cite{murugesan:2001}.
A rápida necessidade de sistemas complexos deixa de lado a preocupação por qualidade à longo prazo,
surgindo a chamada "Crise Web" \cite{murugesan:2001}, uma variação potencialmente mais séria da 
conhecida "Crise de Software" \cite{gibbs:1994}. É importante ressaltar que, embora os sistemas Web
sejam softwares, eles possuem características e requisitos exclusivos \cite{pressman:2011},
o que trouxe a necessidade de uma nova área da Engenharia de Software, a Engenharia Web.

A Engenharia Web pode então ser descrita como a aplicação de princípios e métodos da Engenharia
de Software, adaptados ao desenvolvimento de sistemas Web e suas características específicas 
\cite{beder:2017,murugesan:2001}, com o objetivo de garantir a qualidade dos sistemas Web. 
\cite{olsina:2001} define um conjunto de atributos técnicos que levam a qualidade de um sistema Web,
são eles:

\begin{itemize}
    \item \textbf{Usabilidade}: O sistema deve ser fácil de usar, com uma interface intuitiva e 
        que atenda as necessidades do usuário.
    \item \textbf{Funcionabilidade}: O sistema deve funcionar corretamente, atendendo às características
        do domínio.
    \item \textbf{Eficiência}: O sistema deve fornecer respostas rápidas e precisas.
    \item \textbf{Confiabilidade}: O sistema deve ser confiável, e capaz de se recuperar de erros.
    \item \textbf{Manutenibilidade}: O sistema deve ser fácil de ser corrigido, adaptado e melhorado.
\end{itemize}

A Engenharia Web se baseia em dois conceitos para atingir a qualidade dos sistemas Web: Agilidade e Arcabouço de Processo.
A agilidade é uma abordagem de desenvolvimento de software que se baseia em ciclos curtos de desenvolvimento,
já o Arcabouço de Processo é um conjunto de atividades que devem ser realizadas ao longo de todo o processo
de desenvolvimento do sistema, independente do seu tamanho e complexidade \cite{beder:2017}.


%%%%%%%%% Início de seção. %%%%%%%%%
\section{FrameWeb}
\label{sec-fundteo-frameweb}



%%%%%%%%% Início de seção. %%%%%%%%%
\section{Frameworks}
\label{sec-fundteo-framework}






%%% Páginas finais do documento: bibliografia e anexos. %%%

% Finaliza a parte no bookmark do PDF para que se inicie o bookmark na raiz e adiciona espaço de parte no sumário.
\phantompart

% Marca o início dos elementos pós-textuais.
\postextual

% Referências bibliográficas
\bibliography{bibliografia}





% Fim do documento.
\end{document}
